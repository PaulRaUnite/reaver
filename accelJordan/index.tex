\documentclass[a4paper,11pt]{article}

\usepackage{hyperlatex}
\W\usepackage{sequential}
\W\usepackage{mypanels}
\T\usepackage{fullpage}
\T\usepackage{amsmath}
\T\usepackage{mysymb}
\T\usepackage[english]{babel}
\usepackage{xspace}
\usepackage{color}


\T\usepackage{pstricks,pst-node,pst-tree,pstcol}
\T\usepackage{graphics}
\T\usepackage{url}
\W\newcommand{\url}[1]{\xlink{link}{#1}}


\T\renewcommand{\baselinestretch}{1.02}
\T\setlength{\parindent}{0pt}

\newcommand{\lustre}{\xlink{LUSTRE}{http://www-verimag.imag.fr/SYNCHRONE/}\xspace}

\definecolor{prog}{rgb}{0,0.4,0.1}
\definecolor{progkeyword}{rgb}{0.8,0,0}
\definecolor{token}{rgb}{0,0,1}
\definecolor{emph}{rgb}{0,0.1,0.6}
\definecolor{example}{rgb}{.8,0,1}
\newcommand{\pkw}[1]{\textcolor{progkeyword}{#1}}
\newcommand{\pco}[1]{\textcolor{black}{-- #1}}
\W\renewcommand{\emph}[1]{\textcolor{emph}{#1}}


\newcommand{\sx}[1]{\textcolor{prog}{\texttt{<}#1\texttt{>}}}
\newcommand{\kw}[1]{\textcolor{progkeyword}{\textbf{#1}}}
\newcommand{\tk}[1]{\textcolor{token}{\texttt{#1}}}
\newcommand{\sor}{\texttt{~|~}}

\newenvironment{prog}
{\begin{example}\begin{small}\color{prog}}
{\end{small}\end{example}\par}

\newenvironment{nbac}
{\begin{example}\begin{small}\color{example}}
{\end{small}\end{example}}

\newenvironment{syntax}
{
\T\medskip
\begin{tabular}{rcll}
}
{
\end{tabular}
\T\par\medskip
}

\newcommand{\framenode}[2]{\rnode{#1}{\psframebox{#2}}}
\newcommand{\defnode}[2]{\framenode{#1}{{\Large $\scriptstyle #2$}}}
\newcommand{\astputa}[1]{\naput{\Large $\scriptstyle #1$}}
\newcommand{\astputb}[1]{\nbput{\Large $\scriptstyle #1$}}
\newcommand{\para}[1]{\begin{tabular}{@{\hspace*{-0.2em}}c@{\hspace*{-0.2em}}}#1\end{tabular}}

\setcounter{htmldepth}{2}

%\htmladdress{\xlink{Bertrand Jeannet}{../../index.html}, \today}

\title{Experiments on abstract acceleration of linear loops using Jordan normal forms}
%\author{Bertrand Jeannet}
%\date{October 25, 2006}
\date{June 2012}
%\xmlattributes{body}{bgcolor="#ffffe6"}


\begin{document}

%\xlink{Up}{../index.html}
\maketitle

\begin{center}
\begin{gif}[][130][130]{seqn1}\boldmath
  \vspace*{2ex}

  \color{black}
  \hspace*{8em}
  $
  \psset{arrows=->}
  \begin{psmatrix}[mnode=none,rowsep=10ex,colsep=10em]
    \ovalnode{linit}{linit} \\
    \ovalnode{heat}{heat} \\
    \ovalnode{noheat}{noheat}
  \end{psmatrix}
  \ncline{linit}{heat}\nbput{
    \begin{array}{c}
      T_e\seq 14 \wedge 16\sleq T\sleq 17 \; ?\;  \\
      t := 0
      \end{array}
    }
    \ncarc[arcangle=20]{heat}{noheat}\naput{t\sgeq 22 \; ?\;  t:=0}
    \ncarc[arcangle=20]{noheat}{heat}\naput{t\sleq 18 \; ?\;  t:=0}
    \nccurve[angleA=30,angleB=-30,ncurv=4]{heat}{heat}
    \naput{
      x\sleq 22 \; ?\; \left\{
      \begin{array}{@{}c@{\,}c@{\,}l@{}}
	T' &:=& \frac{15}{16} T - \frac{1}{16} T_e + 1 \\
	t' &:=& t+1
      \end{array}
      \right.
    }
    \nccurve[angleA=30,angleB=-30,ncurv=4]{noheat}{noheat}
    \naput{
      x\sgeq 18 \; ?\; \left\{
      \begin{array}{@{}c@{\,}c@{\,}l@{}}
	T' &:=& \frac{15}{16} T - \frac{1}{16} T_e  \\
	t' &:=& t+1
      \end{array}
      \right.
    }
    $
    \hspace*{12em}~
    \vspace*{2ex}

\end{gif}
\end{center}
\begin{center}
\begin{gif}[][130][130]{seqn2}\boldmath
  \vspace*{2ex}

  \color{black}
  \hspace*{8em}
  $
  \psset{arrows=->}
  \begin{psmatrix}[mnode=none,rowsep=10ex,colsep=10em]
    \ovalnode{linit}{linit} \\
    \ovalnode{heat}{heat} \\
    \ovalnode{noheat}{noheat}
  \end{psmatrix}
  \ncline{linit}{heat}\nbput{
    \begin{array}{c}
      T_e\seq 14 \wedge 16\sleq T\sleq 17 \; ?\;  \\
      t := 0
      \end{array}
    }
    \ncarc[arcangle=20]{heat}{noheat}\naput{t\sgeq 22 \; ?\;  t:=0}
    \ncarc[arcangle=20]{noheat}{heat}\naput{t\sleq 18 \; ?\;  t:=0}
    \nccurve[angleA=30,angleB=-30,ncurv=4]{heat}{heat}
    \naput{
      x\sleq 22 \; ?\; \left\{
      \begin{array}{@{}c@{\,}c@{\,}l@{}}
	T' &:=& \frac{15}{16} T - \frac{1}{16} T_e + 1 \\
	t' &:=& t+1
      \end{array}
      \right.
    }
    \nccurve[angleA=30,angleB=-30,ncurv=4]{noheat}{noheat}
    \naput{
      x\sgeq 18 \; ?\; \left\{
      \begin{array}{@{}c@{\,}c@{\,}l@{}}
	T' &:=& \frac{15}{16} T - \frac{1}{16} T_e  \\
	t' &:=& t+1
      \end{array}
      \right.
    }
    $
    \hspace*{12em}~
    \vspace*{2ex}

\end{gif}
\end{center}
\begin{center}
\begin{gif}[][130][130]{seqn3}\boldmath
  \vspace*{2ex}

  \color{black}
  \hspace*{8em}
  $
  \psset{arrows=->}
  \begin{psmatrix}[mnode=none,rowsep=10ex,colsep=10em]
    \ovalnode{linit}{linit} \\
    \ovalnode{heat}{heat} \\
    \ovalnode{noheat}{noheat}
  \end{psmatrix}
  \ncline{linit}{heat}\nbput{
    \begin{array}{c}
      T_e\seq 14 \wedge 16\sleq T\sleq 17 \; ?\;  \\
      t := 0
      \end{array}
    }
    \ncarc[arcangle=20]{heat}{noheat}\naput{t\sgeq 22 \; ?\;  t:=0}
    \ncarc[arcangle=20]{noheat}{heat}\naput{t\sleq 18 \; ?\;  t:=0}
    \nccurve[angleA=30,angleB=-30,ncurv=4]{heat}{heat}
    \naput{
      x\sleq 22 \; ?\; \left\{
      \begin{array}{@{}c@{\,}c@{\,}l@{}}
	T' &:=& \frac{15}{16} T - \frac{1}{16} T_e + 1 \\
	t' &:=& t+1
      \end{array}
      \right.
    }
    \nccurve[angleA=30,angleB=-30,ncurv=4]{noheat}{noheat}
    \naput{
      x\sgeq 18 \; ?\; \left\{
      \begin{array}{@{}c@{\,}c@{\,}l@{}}
	T' &:=& \frac{15}{16} T - \frac{1}{16} T_e  \\
	t' &:=& t+1
      \end{array}
      \right.
    }
    $
    \hspace*{12em}~
    \vspace*{2ex}

\end{gif}
\end{center}

%******************************************************************************
\section{Introduction}
\label{sec:intro}
%******************************************************************************

\xlink{Research report with appendix}{mainappendix.pdf}


%******************************************************************************
\section{Thermostat experiment}
\label{sec:thermostat}
%******************************************************************************

%==============================================================================
\subsection{The model}
%==============================================================================

We start from the following differential linear hybrid automaton.

\begin{center}
\begin{gif}[][130][130]{eqn1}\boldmath
  \vspace*{2ex}

  \color{black}
  \hspace*{8em}
  $
  \psset{arrows=->}
  \begin{psmatrix}[mnode=none,rowsep=10ex,colsep=10em]
    \pnode{linit} \\
    \ovalnode{heat}{\dot{T}=1-\frac{1}{16}(T-T_e)} \\
    \ovalnode{noheat}{\dot{T}=-\frac{1}{16}(T-T_e)}
  \end{psmatrix}
  \ncline{linit}{heat}\nbput{
    \begin{array}{c}
      T_e\seq 14 \wedge 16\sleq T\sleq 17 \; ?\;  \\
      t := 0
      \end{array}
    }
    \ncarc[arcangle=20]{heat}{noheat}\naput{t\sgeq 22 \; ?\;  t:=0}
    \ncarc[arcangle=20]{noheat}{heat}\naput{t\sleq 18 \; ?\;  t:=0}
    $
    \vspace*{2ex}

\end{gif}
\end{center}
The variables are the room temperature ($T$), the external
temperature ($T_e$) and the time clock ($t$). The upper location
corresponds to the heating mode, the lower one to the cooling
mode.  The goal is to maintain the room temperature around $20$
degrees. If this temperature is less than $18$, the heating is
switched on. If it exceeds $22$, heating is switched off (there is
an hysteresis).

This small system is first (very naively) discretized using a sampling time of 1 second.
\begin{center}
\begin{gif}[][130][130]{eqn2}\boldmath
  \vspace*{2ex}

  \color{black}
  \hspace*{8em}
  $
  \psset{arrows=->}
  \begin{psmatrix}[mnode=none,rowsep=10ex,colsep=10em]
    \ovalnode{linit}{linit} \\
    \ovalnode{heat}{heat} \\
    \ovalnode{noheat}{noheat}
  \end{psmatrix}
  \ncline{linit}{heat}\nbput{
    \begin{array}{c}
      T_e\seq 14 \wedge 16\sleq T\sleq 17 \; ?\;  \\
      t := 0
      \end{array}
    }
    \ncarc[arcangle=20]{heat}{noheat}\naput{t\sgeq 22 \; ?\;  t:=0}
    \ncarc[arcangle=20]{noheat}{heat}\naput{t\sleq 18 \; ?\;  t:=0}
    \nccurve[angleA=30,angleB=-30,ncurv=4]{heat}{heat}
    \naput{
      x\sleq 22 \; ?\; \left\{
      \begin{array}{@{}c@{\,}c@{\,}l@{}}
	T' &:=& \frac{15}{16} T - \frac{1}{16} T_e + 1 \\
	t' &:=& t+1
      \end{array}
      \right.
    }
    \nccurve[angleA=30,angleB=-30,ncurv=4]{noheat}{noheat}
    \naput{
      x\sgeq 18 \; ?\; \left\{
      \begin{array}{@{}c@{\,}c@{\,}l@{}}
	T' &:=& \frac{15}{16} T - \frac{1}{16} T_e  \\
	t' &:=& t+1
      \end{array}
      \right.
    }
    $
    \hspace*{12em}~
    \vspace*{2ex}

\end{gif}
\end{center}
and then transformed for acceleration:
\begin{center}
\begin{gif}[][120][120]{eqn3}\boldmath
  \vspace*{2ex}

  \color{black}
  \hspace*{8em}
  $
  \psset{arrows=->}
  \begin{psmatrix}[mnode=none,rowsep=10ex,colsep=2.5em]
    \ovalnode{linit}{linit} \\
    \ovalnode{heat}{heat\_start} &&& \ovalnode{heata}{heat} & \ovalnode{heatb}{heate} \\
    \ovalnode{noheatb}{noheate} & \ovalnode{noheata}{noheat} &&& \ovalnode{noheat}{noheat\_start}
  \end{psmatrix}
  \ncline{linit}{heat}\nbput{
    \begin{array}{c}
      T_e\seq 14 \wedge 16\sleq T\sleq 17 \; ?\;  \\
      t := 0
    \end{array}
  }
  \ncline{heat}{heata}\naput{
    \left(
      x\sleq 22 \; ?\; \left\{
	\begin{array}{@{}c@{\,}c@{\,}l@{}}
	  T' &:=& \frac{15}{16} T - \frac{1}{16} T_e + 1 \\
	  t' &:=& t+1
	\end{array}
      \right.
    \right)^*
  }
  \ncline{heata}{heatb}\naput{t\sgeq 22\; ?}
  \ncline{heatb}{noheat}\naput{t := 0}
  \ncline{noheat}{noheata}\naput{
    \left(
      x\sgeq 18 \; ?\; \left\{
	\begin{array}{@{}c@{\,}c@{\,}l@{}}
	  T' &:=& \frac{15}{16} T - \frac{1}{16} T_e \\
	  t' &:=& t+1
	\end{array}
      \right.
    \right)^*
  }
  \ncline{noheata}{noheatb}\naput{t\sleq 18\; ?}
  \ncline{noheatb}{heat}\naput{t := 0}
  $
  \vspace*{5ex}

\end{gif}
\end{center}

The corresponding input model of our prototype is
\xlink{thermostat.lts}{thermostat.lts} In it, the keyword
``jordan'' give the decomposition of the assignement $M$ into
$S^{-1}JS$ where $J$ is the real Jordan form of $M$. The last,
fourth dimension corresponds to the added variable $\xi$ always
equal to 1.

%==============================================================================
\subsection{The analysis}
%==============================================================================

We type the command
\begin{prog}
jordan.byte -debug 1 -print box -dot thermostat1.dot -log 1 thermostat.lts &&
dot -Tps thermostat1.dot >thermostat1.ps
\end{prog}
where the ``-log l'' option indicates that when abstracting
matrices with template polyhedra, we consider the set of template
expressions of the form
\begin{center}
\begin{gif}[][130][130]{eqn4}\boldmath
$\alpha x_i +/- (1-\alpha) x_j$
with
$
\alpha\in\{ \frac{k}{2^\ell} \;|\; 0\leq k\leq 2^\ell \}
$
\end{gif}
\end{center}
If $d$ is the
number of different coefficients of the set of matrices to
abstract, this generates $d(d-1)(2^l-1) + d$ template
expressions to be lower and upper bounded.

The other options are related to the output.

The result can be viewed at the last page of the
\xlink{thermostat1.ps}{thermostat1.ps} file.

Some observations:
\begin{itemize}
\item In each location is displayed the computed invariant, under
  the form of a set of constraints, followed by the bounding box
  of the corresponding polyhedron.

\item Looking at location ``heat1'', one observes that
  temperature $T in [16,22.5]$ in heating mode, and looking at location
  ``noheat1'', one observes that $T in [17.75,22.5]$ in cooling
  mode.

\item Looking at location ``heat2'', the analysis infers that the
  system spents \emph{at least}  $1.18$ and \emph{at most} $9.98$ consecutive seconds in the
  heating mode, whereas looking at location ``noheat2'', the
  analysis infers that the system spents  \emph{at least} $1.41$
  and \emph{at most} $12.99$
  consecutive seconds in this mode.

  The non-zero lower bounds indicates that the hysteresis works:
  the system mode is not switched infinitely often in a finite time.
\end{itemize}

It is instructive to see how these significant numbers are
improved if one increases the precision with the ``-log l''
option. The following table synthesizes our experiments.

\begin{center}
\begin{tabular}{|c|c|cc|cc|cc|}
\hline
l & \#e & \multicolumn{2}{c|}{heating mode} &
\multicolumn{2}{c|}{cooling mode} & running time & file \\
&& T & time & T & time && \\ \hline
1 & 4 & $[16,22.5]$ & $[1.19,9.99]$ & $[17.75,22.5]$ & $[1.41,12.99]$
& & \xlink{thermostat1.ps}{thermostat1.ps} \\
2 & 8 & $[16,22.5]$ & $[1.78,9.96]$ & $[17.75,22.5]$ & $[2.35,12.97]$
& & \xlink{thermostat2.ps}{thermostat2.ps} \\
3 & 16& $[16,22.5]$ & $[2.96,9.92]$ & $[17.75,22.5]$ & $[4.24,12.93]$
& & \xlink{thermostat3.ps}{thermostat3.ps} \\
4 & 32& $[16,22.5]$ & $[5.33,9.82]$ & $[17.75,22.5]$ & $[8.00,12.86]$
& & \xlink{thermostat4.ps}{thermostat4.ps} \\
5 & 64& $[16,22.5]$ & $[5.33,9.82]$ & $[17.75,22.5]$ & $[10.71,12.86]$
& & \xlink{thermostat5.ps}{thermostat5.ps} \\
6 & 128& $[16,22.5]$ & $[6.25,9.76]$ & $[17.75,22.5]$ & $[10.71,12.80]$
& & \xlink{thermostat6.ps}{thermostat6.ps} \\
8 & 512& $[16,22.5]$ & $[6.28,9.76]$ & $[17.75,22.5]$ & $[10.72,12.79]$
& & \xlink{thermostat8.ps}{thermostat8.ps} \\
\hline
\end{tabular}

\begin{tabular}{lll}
  l &:& corresponds to ``-log l'' option \\
  \#e & : & number of template expressions on the coefficients of
  Jordan normal form \\
  && (related to the number ``l'' and the number of
different coefficient of the Jordan normal form) \\
T &:& regulated temperature \\
time &:& time spent in each mode
\end{tabular}

\end{center}

The temperature intervals do not change, but the lower bound for
time improves a lot up to $l=4$, and slightly afterwards.

In the sequel, in the Postscript files we display only the
bounding boxed of the inferred polyhedra (used option ``-print box''
instead of ``-print boxpoly'').

%******************************************************************************
\section{Parabola, cubic and exponential behaviours}
\label{sec:parabola}
%******************************************************************************

\subsection{Parabola}

We study an integrator put under different initial conditions and
guards, formalized by the following automaton:
\begin{center}
\begin{gif}[][130][130]{eqnpara1}\boldmath
  \vspace*{10ex}

  \color{black}
  \hspace*{8em}
  $
  \psset{arrows=->}
  \begin{psmatrix}[mnode=none,rowsep=12ex,colsep=4em]
    &&& \ovalnode{para}{para} & \ovalnode{parae}{parae} \\
    \ovalnode{linit}{linit} &&& \ovalnode{paray}{paray} & \ovalnode{paraye}{paraye} \\
    \ovalnode{l}{l} &&& \ovalnode{parax}{parax} & \ovalnode{paraxe}{paraxe} \\
    &&& \ovalnode{paraxy}{paraxy} & \ovalnode{paraxye}{paraxye} \\
    &&& \ovalnode{paraxyy}{paraxyy} & \ovalnode{paraxyye}{paraxyye}
  \end{psmatrix}
  \ncline{linit}{l}\nbput{
    \begin{array}{c}
    x\seq 0\wedge y\seq 0 \; ?\\
    t := 0
  \end{array}
  }
  \nccurve[angleA=80,angleB=-180,ncurvB=1.3]{l}{para}\naput[npos=0.85]{
    \left(
      \mathsf{true} \; ? \left\{
	\begin{array}{@{}c@{\,}c@{\,}l@{}}
	  x' &:=& x+y\\
	  y' &:=& y+1\\
	  t' &:=& t+1
	\end{array}
      \right.
    \right)^*
  }
  \ncline{para}{parae}\naput{\mathsf{true}\; ?}
  \nccurve[angleA=80,angleB=-180,ncurvB=1.3]{l}{paray}\naput[npos=0.8]{
    \left(
      y\sleq 7 \; ? \left\{
	\begin{array}{@{}c@{\,}c@{\,}l@{}}
	  x' &:=& x+y\\
	  y' &:=& y+1\\
	  t' &:=& t+1
	\end{array}
      \right.
    \right)^*
  }
  \ncline{paray}{paraye}\naput{y\sgeq 8\; ?}
  \nccurve[angleA=0,angleB=-180,ncurvB=1.3]{l}{parax}\naput[npos=0.75]{
    \left(
      x\sleq 30 \; ? \left\{
	\begin{array}{@{}c@{\,}c@{\,}l@{}}
	  x' &:=& x+y\\
	  y' &:=& y+1\\
	  t' &:=& t+1
	\end{array}
      \right.
    \right)^*
  }
  \ncline{parax}{paraxe}\naput{x\sgeq 31\; ?}
  \nccurve[angleA=-80,angleB=-180,ncurvB=1.3]{l}{paraxy}\naput[npos=0.8]{
    \left(
      x\sadd y\sleq 30 \; ? \left\{
	\begin{array}{@{}c@{\,}c@{\,}l@{}}
	  x' &:=& x+y\\
	  y' &:=& y+1\\
	  t' &:=& t+1
	\end{array}
      \right.
    \right)^*
  }
  \ncline{paraxy}{paraxye}\naput{x\sadd y\sgeq 31\; ?}
  \nccurve[angleA=-80,angleB=-180,ncurvB=1.3]{l}{paraxyy}\naput[npos=0.85]{
    \left(
      x\ssub 2y\sleq 30 \; ? \left\{
	\begin{array}{@{}c@{\,}c@{\,}l@{}}
	  x' &:=& x+y\\
	  y' &:=& y+1\\
	  t' &:=& t+1
	\end{array}
      \right.
    \right)^*
  }
  \ncline{paraxyy}{paraxyye}\naput{x\ssub 2y\sgeq 31\; ?}
  $
  \vspace*{5ex}

\end{gif}
\end{center}
The trajectories are of the form ``x = y*y + a*y + b''.
%
The purpose of the different guards is to demonstrate the ability
of the technique to perform non-linear deductions.


\begin{itemize}
\item The analysis results of \xlink{parabola\_i0.lts}{parabola\_i0.lts} can be seen here.
\begin{center}
\begin{tabular}{|c|c|c|c|}
\hline
l & \#e & running time & file \\
1 & 4 & & \xlink{parabola\_i0\_l1.ps}{parabola\_i0\_l1.ps} \\
2 & 8 & & \xlink{parabola\_i0\_l2.ps}{parabola\_i0\_l2.ps} \\
4 & 32 & & \xlink{parabola\_i0\_l4.ps}{parabola\_i0\_l4.ps} \\
\hline
\end{tabular}

\begin{tabular}{lll}
  l &:& corresponds to ``-log l'' option \\
  \#e & : & number of template expressions on the coefficients of
  Jordan normal form \\
  && (related to the number ``l'' and the number of
different coefficient of the Jordan normal form)
\end{tabular}
\end{center}

\item If we replace the initial condition with ``0 <= x <= 2'' and
  ``0 <= y <= 2'',
we obtain these results:
\begin{center}
\begin{tabular}{|c|c|c|c|}
\hline
l & \#e & running time & file \\
1 & 4 & & \xlink{parabola\_i1\_l1.ps}{parabola\_i1\_l1.ps} \\
2 & 8 & & \xlink{parabola\_i1\_l2.ps}{parabola\_i1\_l2.ps} \\
4 & 32 & & \xlink{parabola\_i1\_l4.ps}{parabola\_i1\_l4.ps} \\
\hline
\end{tabular}
\end{center}

\item If we replace the initial condition with ``-2 <= x <= 2'' and
  ``0-2 <= y <= 2'',
we obtain these results:
\begin{center}
\begin{tabular}{|c|c|c|c|}
\hline
l & \#e & running time & file \\
1 & 4 & & \xlink{parabola\_i2\_l1.ps}{parabola\_i2\_l1.ps} \\
2 & 8 & & \xlink{parabola\_i2\_l2.ps}{parabola\_i2\_l2.ps} \\
4 & 32 & & \xlink{parabola\_i2\_l4.ps}{parabola\_i2\_l4.ps} \\
\hline
\end{tabular}
\end{center}
\end{itemize}

\subsection{Cubic behaviour}

Let add now a variable, in order to get a cubic behaviour.
\begin{center}
\begin{gif}[][130][130]{eqncubic1}\boldmath
  \vspace*{12ex}

  \color{black}
  \hspace*{8em}
  $
  \psset{arrows=->}
  \begin{psmatrix}[mnode=none,rowsep=12ex,colsep=4em]
    &&& \ovalnode{cubic}{cubic} & \ovalnode{cubice}{cubice} \\
    \ovalnode{linit}{linit} &&& \ovalnode{cubicy}{cubicy} & \ovalnode{cubicye}{cubicye} \\
    \ovalnode{l}{l} &&& \ovalnode{cubicx}{cubicx} & \ovalnode{cubicxe}{cubicxe} \\
    &&& \ovalnode{cubicxy}{cubicxy} & \ovalnode{cubicxye}{cubicxye} \\
    &&& \ovalnode{cubicxyy}{cubicxyy} & \ovalnode{cubicxyye}{cubicxyye}
  \end{psmatrix}
  \ncline{linit}{l}\nbput{
    \begin{array}{c}
    x\seq 0\wedge y\seq 0 \wedge z\seq 0\; ?\\
    t := 0
  \end{array}
  }
  \nccurve[angleA=80,angleB=-180,ncurvB=1.3]{l}{cubic}\naput[npos=0.85]{
    \left(
      \mathsf{true} \; ? \left\{
	\begin{array}{@{}c@{\,}c@{\,}l@{}}
	  x' &:=& x+y\\
	  y' &:=& y+z\\
	  z' &:=& z+1\\
	  t' &:=& t+1
	\end{array}
      \right.
    \right)^*
  }
  \ncline{cubic}{cubice}\naput{\mathsf{true}\; ?}
  \nccurve[angleA=80,angleB=-180,ncurvB=1.3]{l}{cubicy}\naput[npos=0.8]{
    \left(
      y\sleq 7 \; ? \left\{
	\begin{array}{@{}c@{\,}c@{\,}l@{}}
	  x' &:=& x+y\\
	  y' &:=& y+z\\
	  z' &:=& z+1\\
	  t' &:=& t+1
	\end{array}
      \right.
    \right)^*
  }
  \ncline{cubicy}{cubicye}\naput{y\sgeq 8\; ?}
  \nccurve[angleA=0,angleB=-180,ncurvB=1.3]{l}{cubicx}\naput[npos=0.75]{
    \left(
      x\sleq 30 \; ? \left\{
	\begin{array}{@{}c@{\,}c@{\,}l@{}}
	  x' &:=& x+y\\
	  y' &:=& y+z\\
	  z' &:=& z+1\\
	  t' &:=& t+1
	\end{array}
      \right.
    \right)^*
  }
  \ncline{cubicx}{cubicxe}\naput{x\sgeq 31\; ?}
  \nccurve[angleA=-80,angleB=-180,ncurvB=1.3]{l}{cubicxy}\naput[npos=0.8]{
    \left(
      x\sadd y\sleq 30 \; ? \left\{
	\begin{array}{@{}c@{\,}c@{\,}l@{}}
	  x' &:=& x+y\\
	  y' &:=& y+z\\
	  z' &:=& z+1\\
	  t' &:=& t+1
	\end{array}
      \right.
    \right)^*
  }
  \ncline{cubicxy}{cubicxye}\naput{x\sadd y\sgeq 31\; ?}
  \nccurve[angleA=-80,angleB=-180,ncurvB=1.3]{l}{cubicxyy}\naput[npos=0.85]{
    \left(
      x\ssub 2y\sleq 30 \; ? \left\{
	\begin{array}{@{}c@{\,}c@{\,}l@{}}
	  x' &:=& x+y\\
	  y' &:=& y+z\\
	  z' &:=& z+1\\
	  t' &:=& t+1
	\end{array}
      \right.
    \right)^*
  }
  \ncline{cubicxyy}{cubicxyye}\naput{x\ssub 2y\sgeq 31\; ?}
  $
  \vspace*{5ex}

\end{gif}
\end{center}
The trajectories are now of the form ``x=z*z*z + a*z*z + b*z+c''
and ``y=z*z + d*z + e''.

\begin{itemize}
\item The analysis results  of \xlink{cubic\_i0.lts}{cubic\_i0.lts} can be seen here.
\begin{center}
\begin{tabular}{|c|c|c|c|}
\hline
l & \#e & running time & file \\
1 & 9 & & \xlink{cubic\_i0\_l1.ps}{cubic\_i0\_l1.ps} \\
2 & 27 & & \xlink{cubic\_i0\_l2.ps}{cubic\_i0\_l2.ps} \\
4 & 93 & & \xlink{cubic\_i0\_l4.ps}{cubic\_i0\_l4.ps} \\
\hline
\end{tabular}
\end{center}

\item If we replace the initial condition with
``0 <= x <= 2'' and ``0 <= y <= 2'' and ``0 <= z <= 2'',
we obtain these results:
\begin{center}
\begin{tabular}{|c|c|c|c|}
\hline
l & \#e & running time & file \\
1 & 9 & & \xlink{cubic\_i1\_l1.ps}{cubic\_i1\_l1.ps} \\
2 & 27 & & \xlink{cubic\_i1\_l2.ps}{cubic\_i1\_l2.ps} \\
4 & 93 & & \xlink{cubic\_i1\_l4.ps}{cubic\_i1\_l4.ps} \\
\hline
\end{tabular}
\end{center}

\item If we replace the initial condition with
``-2 <= x <= 2'' and ``-2 <= y <= 2'' and ``-2 <= z <= 2'',
we obtain these results:
\begin{center}
\begin{tabular}{|c|c|c|c|}
\hline
l & \#e & running time & file \\
1 & 9 & & \xlink{cubic\_i2\_l1.ps}{cubic\_i2\_l1.ps} \\
2 & 27 & & \xlink{cubic\_i2\_l2.ps}{cubic\_i2\_l2.ps} \\
4 & 93 & & \xlink{cubic\_i2\_l4.ps}{cubic\_i2\_l4.ps} \\
\hline
\end{tabular}
\end{center}
\end{itemize}

\subsection{Exponential behaviour}
We also analyse the example \xlink{exp\_div.lts}{exp\_div.lts}:
\begin{center}
\begin{tabular}{|c|c|c|c|}
\hline
l & \#e & running time & file \\
1 & 4 & & \xlink{exp\_div\_l1.ps}{exp\_div\_l1.ps} \\
2 & 8 & & \xlink{exp\_div\_l2.ps}{exp\_div\_l2.ps} \\
3 & 16 & & \xlink{exp\_div\_l3.ps}{exp\_div\_l3.ps} \\
\hline
\end{tabular}
\end{center}

%******************************************************************************
\section{Damped oscillator}
\label{sec:oscillator}
%******************************************************************************

We consider a damped oscillator (a pendulum in the linear approximation), ruled by the differential
equation

\begin{gif}[][130][130]{eqn5}\boldmath
$\ddot{\theta} + 2k\dot{\theta} + \omega_0^2\theta=0$
where $2k$ is the damping factor and $\omega_0$ the pulsation when $k=0$.

If $k<\omega_0$, the solution is of the form
$\theta(t)=\theta_0 e^{-kt}cos(\omega t+\varphi)$ with
$\omega=\sqrt{\omega_0^2-k^2}$.

If $k=\omega_0$, the solution is of the form $\theta(t)=(\theta_0
+theta_0' t) e^{-kt}$.

If $k>\omega_0$, the solution is of the form
$\theta(t)=\mu_1 e^{(-k-\sqrt{k^2-\omega^2})t} + \mu_2 e^{(-k+\sqrt{k^2-\omega^2})t}$
\end{gif}

\par

We naively discretized this equation (using a sampling time of
one second) with
\begin{center}
\begin{gif}[][130][130]{eqn6}\boldmath
$
\begin{array}{rcl}
\theta_p' &=& \theta_p + \theta_v \\
\theta_v' &=& -\omega^2\theta_p + (1-2k) \theta_v
\end{array}
$
\end{gif}
\end{center}
Choosing $\omega=1/8$ and various values for $k$, and
choosing initial condition $\theta_p=8$ and $\theta_v=0$, we have
the following system \xlink{oscillator\_i0.lts}{oscillator\_i0.lts}.

\begin{itemize}
\item The analysis results can be seen here.
\begin{center}
\begin{tabular}{|c|c|cc|cc|cc|cc|cc|cc|c|c|}
\hline
l & \#e & \multicolumn{2}{|c}{$k=3/4$} &
\multicolumn{2}{|c}{$k=1/4$} & \multicolumn{2}{|c}{$k=1/8$} &
\multicolumn{2}{|c}{$k=1/16$} & \multicolumn{2}{|c}{$k=1/64$} &
\multicolumn{2}{|c}{$k=0$} &
running time & file \\
&&
$\theta_p$ & $\theta_v$ &
$\theta_p$ & $\theta_v$ &
$\theta_p$ & $\theta_v$ &
$\theta_p$ & $\theta_v$ &
$\theta_p$ & $\theta_v$ &
$\theta_p$ & $\theta_v$
\\ \hline
1 & 4 &
$[0,8.0]$ & $[-0.125,0]$ &
$[0,8.264]$ & $[-0.248,0]$ &
$[0,10.217]$ & $[-0.402,0]$ &
$[-1.818,8.262]$ & $[-0.625,0.128]$ &
$[-6.580,8]$ & $[-0.956,0.785]$ &
$[-\infty,\infty]$ & $[-\infty,\infty]$
& & \xlink{oscillator\_i0\_l1.ps}{oscillator\_i0\_l1.ps} \\
2 & 8 &
$[0,8.0]$ & $[-0.125,0]$ &
$[0,8.184]$ & $[-0.233,0]$ &
$[0,9.286]$ & $[-0.394,0]$ &
$[-1.680,8.090]$ & $[-0.600,0.123]$ &
$[-6.580,8]$ & $[-0.924,0.760]$ &
$[-\infty,\infty]$ & $[-\infty,\infty]$
& & \xlink{oscillator\_i0\_l2.ps}{oscillator\_i0\_l2.ps} \\
4 & 32&
$[0,8.0]$ & $[-0.125,0]$ &
$[0,8.0]$ & $[-0.231,0]$ &
$[0,8.08]$ & $[-0.393,0]$ &
$[-1.648,8.0]$ & $[-0.592,0.122]$ &
$[-6.580,8]$ & $[-0.915,0.751]$ &
$[-\infty,\infty]$ & $[-\infty,\infty]$
& & \xlink{oscillator\_i0\_l4.ps}{oscillator\_i0\_l4.ps} \\
\hline
\end{tabular}

\begin{tabular}{lll}
  l &:& corresponds to ``-log l'' option \\
  \#e & : & number of template expressions on the coefficients of
  Jordan normal form \\
  && (related to the number ``l'' and the number of
different coefficient of the Jordan normal form) \\
  k &:& damping factor \\
  $theta_p$ &:& minimum and maximum angular position in discrete
  trajectories
  \\
  $theta_v$ &:& minimum and maximum angular speed in discrete
  trajectories
\end{tabular}
\end{center}

\item If we now choose the initial condition $\theta_p=8$ and
$\theta_v=8$, we have these results:
\begin{center}
\begin{tabular}{|c|c|cc|cc|cc|cc|cc|cc|c|c|}
\hline
l & \#e & \multicolumn{2}{|c}{$k=3/4$} &
\multicolumn{2}{|c}{$k=1/4$} & \multicolumn{2}{|c}{$k=1/8$} &
\multicolumn{2}{|c}{$k=1/16$} & \multicolumn{2}{|c}{$k=1/64$} &
\multicolumn{2}{|c}{$k=0$} &
running time & file \\
&&
$\theta_p$ & $\theta_v$ &
$\theta_p$ & $\theta_v$ &
$\theta_p$ & $\theta_v$ &
$\theta_p$ & $\theta_v$ &
$\theta_p$ & $\theta_v$ &
$\theta_p$ & $\theta_v$
\\ \hline
4 & 32&
$[0,16.0]$ & $[-4.125,8]$ &
$[0,16.099]$ & $[-0.25,8]$ &
$[0,31.416]$ & $[-1.560,8]$ &
$[-8.742,42.555]$ & $[-3.144,8]$ &
$[-49.598,60.421]$ & $[-6.901,8]$ &
$[-\infty,\infty]$ & $[-\infty,\infty]$
& & \xlink{oscillator\_i1\_l4.ps}{oscillator\_i1\_l4.ps} \\
\hline
\end{tabular}
\end{center}
\end{itemize}

Some comments:
\begin{itemize}
\item If $k=0$, the discretized version diverges. Indeed the
  Jordan matrix have an eigenvalue with a norm strictly greater
  than 1.
\item When $k$ is larger, the oscillations become smaller both in
  magnitude and maximal speed.
\item When the initial speed is zero, the analysis discovers that when
  $k>\omega$, the position cannot be negative.
\end{itemize}

%******************************************************************************
\section{Inverted pendulum}
\label{sec:invpendulum}
%******************************************************************************

We consider an inverted pendulum
\xlink{inv\_pendulum.lts}{inv\_pendulum.lts} together with its PID
controller. The variables are ``pos'': position of the mobile,
``speed'': its speed, ``ang'': angle of the pendulum (in degrees),
and ``angDer'': angular speed (in degrees per sec). The initial
condition is ``-1 <= ang <= 1'' and ``pos=speed=angDer=0''.

The analysis shows that the controller ensures the stability of
the system and delivers lower and upper bounds on the different
quantities (internally it uses convex polyhedra to relate them).

\begin{center}
\begin{tabular}{|c|c|c|c|}
\hline
l & \#e & running time & file \\
0 & 4 & & \xlink{inv\_pendulum\_l0.ps}{inv\_pendulum\_l0.ps} \\
1 & 16 & & \xlink{inv\_pendulum\_l1.ps}{inv\_pendulum\_l1.ps} \\
2 & 40 & & \xlink{inv\_pendulum\_l2.ps}{inv\_pendulum\_l2.ps} \\
3 & 88 & & \xlink{inv\_pendulum\_l3.ps}{inv\_pendulum\_l3.ps} \\
4 & 188 & & \xlink{inv\_pendulum\_l4.ps}{inv\_pendulum\_l4.ps} \\
\hline
\end{tabular}
\end{center}



%******************************************************************************
\section{Damped oscillator with two modes}
\label{sec:hoscillator}
%******************************************************************************

We saw that the (naive) discretized version of the undamped
oscillator diverges. We now consider an oscillator where only the
central part is damped:
\begin{center}\tiny
\begin{gif}[][160][160]{eqn7}\boldmath
  \vspace*{2ex}

  \color{black}
  \hspace*{12em}
  $
  \psset{arrows=->}
  \begin{psmatrix}[mnode=none,rowsep=10ex,colsep=3.5em]
    &\ovalnode{linit}{linit} \\
    &\ovalnode{freep}{freep} \\
    \ovalnode{dampedp}{dampedp} && \ovalnode{dampedn}{dampedn} \\
    & \ovalnode{freen}{freen}
  \end{psmatrix}
  \ncline{linit}{freep}\nbput{
    \begin{array}{c}
      \theta_p=10 \wedge \theta_v=0 \; ?\; \\
    \end{array}
  }
  \ncarc[arcangle=-15]{freep}{dampedp}\nbput[npos=0.35,labelsep=0pt]{\theta_p\sleq 4 \; ?}
  \ncarc[arcangle=-15]{dampedp}{freep}\nbput[npos=0.35,labelsep=0pt]{\theta_p\sgeq 4 \; ?}
  \ncarc[arcangle=15]{dampedn}{freep}\naput[npos=0.35,labelsep=0pt]{\theta_p\sgeq 4 \; ?}
  \ncarc[arcangle=-15]{dampedp}{freen}\nbput[npos=0.35,labelsep=0pt]{\theta_p\sleq -4 \; ?}
  \ncarc[arcangle=15]{freen}{dampedn}\naput[npos=0.35,labelsep=0pt]{\theta_p\sgeq -4 \; ?}
  \ncarc[arcangle=15]{dampedn}{freen}\naput[npos=0.35,labelsep=0pt]{\theta_p\sleq -4 \; ?}
  \nccurve[angleA=30,angleB=-30,ncurv=4]{freep}{freep}
  \naput{
    \theta_p\sgeq 4 \; ?\; \left\{
      \begin{array}{@{}c@{\,}c@{\,}l@{}}
	\theta_p' &:=& \theta_p+\theta_v \\
	\theta_v' &:=& -\omega^2\theta_p+\theta_v
      \end{array}
    \right.
  }
  \nccurve[angleA=150,angleB=-150,ncurv=4]{dampedp}{dampedp}
  \nbput{
    -4\sleq \theta_p\sleq 4 \; ?\; \left\{
      \begin{array}{@{}c@{\,}c@{\,}l@{}}
	\theta_p' &:=& \theta_p+\theta_v \\
	\theta_v' &:=& -\omega^2\theta_p+(1-2k)\theta_v
      \end{array}
    \right.
  }
  \nccurve[angleA=30,angleB=-30,ncurv=4]{dampedn}{dampedn}
  \naput{
    -4\sleq \theta_p\sleq 4 \; ?\; \left\{
      \begin{array}{@{}c@{\,}c@{\,}l@{}}
	\theta_p' &:=& \theta_p+\theta_v \\
	\theta_v' &:=& -\omega^2\theta_p+(1-2k)\theta_v
      \end{array}
    \right.
  }
  \nccurve[angleA=30,angleB=-30,ncurv=4]{freen}{freen}
  \naput{
    \theta_p\sleq -4 \; ?\; \left\{
      \begin{array}{@{}c@{\,}c@{\,}l@{}}
	\theta_p' &:=& \theta_p+\theta_v \\
	\theta_v' &:=& -\omega^2\theta_p+\theta_v
      \end{array}
      \right.
    }
    $
    \hspace*{12em}~
    \vspace*{2ex}

\end{gif}
\end{center}

We still add to split the central mode into two modes (kind of
trace partitioning) to prove that the behaviour is bounded,
despite the two extremal modes being unstables.

The actual source files are either
\xlink{oscillation2\_16.lts}{oscillation2\_16.lts} ($k=1/16$) or
\xlink{oscillation2\_32.lts}{oscillation2\_32.lts} ($k=1/32$).

The results can be seen here (we apply widening only ath the 3rd
iteration):
\begin{center}
\begin{tabular}{|c|c|c|c|c|}
\hline
l & \#e & k & running time & file \\
2 & 4 & 1/16 && \xlink{oscillation2\_16\_l2w3.ps}{oscillation2\_16\_l2w3.ps} \\
3 & 8 & 1/16 && \xlink{oscillation2\_16\_l3w3.ps}{oscillation2\_16\_l3w3.ps} \\
4 & 32 & 1/16 && \xlink{oscillation2\_16\_l4w3.ps}{oscillation2\_16\_l4w3.ps} \\
2 & 4 & 1/32 && \xlink{oscillation2\_32\_l2w3.ps}{oscillation2\_32\_l2w3.ps} \\
3 & 8 & 1/32 && \xlink{oscillation2\_32\_l3w3.ps}{oscillation2\_32\_l3w3.ps} \\
4 & 32 & 1/32 && \xlink{oscillation2\_32\_l4w3.ps}{oscillation2\_32\_l4w3.ps} \\
\hline
\end{tabular}
\end{center}

We need $l\geq 3$ to prove global stability. Moreover, if it is
the case and in the case $k=1/16$, we prove that the system cannot exit
from location ``dampedn''.

%******************************************************************************
\section{Convoy Car example}
\label{sec:convoyCar}
%******************************************************************************

\subsection{The example}
In this example, there is leading car followed by one or two cars
behind it, trying to maintain their position at 50m from each
other.

\begin{center}
\begin{gif}[][130][130]{convoyeqn0}\boldmath
  \vspace*{4ex}

  \begin{psmatrix}[rowsep=1ex,colsep=6em,emnode=p,mnode=R]
    & & & & \\
    & \psframebox{car2} & \psframebox{car1} & \psframebox{car0} & \\
    & & & &
  \end{psmatrix}
  \ncline{2,1}{2,2}
  \ncline{2,2}{2,3}
  \ncline{2,3}{2,4}
  \ncline[arrows=->]{2,4}{2,5}
  \ncline[arrows=<->]{1,2}{1,3}\naput{$\mathit{dp}_1$}
  \ncline[arrows=<->]{1,3}{1,4}\naput{$\mathit{dp}_0$}
  \nput{-90}{3,2}{$x_2=p_2$}
  \nput{-90}{3,3}{$x_1=p_1$}
  \nput{-90}{3,4}{$x_0=p_0$}
  \vspace*{4ex}
\end{gif}
\end{center}

The equations are:
\begin{center}
\begin{gif}[][130][130]{convoyeqn1}\boldmath
$
x_i'' = \lambda(x'_i-x'_{i+1}) + \lambda_2(x_i-x_{i+1}-50)
$
\end{gif}
\end{center}
for $i in [1,N-1]$, with $x''_i$, $x'_i$ and $x_i$ the
acceleration, speed and position of vehicle $i$. $i=0$
corresponds to the leading car, $N$ being the toal number of cars.

To ease the analysis of the results, we reformulate the
problems by considering the deltas of positions and speeds,
$\mathit{dv}_i=x'_i-x'_{i+1}$ and $\mathit{dp}_i = x_i-x_{i+1}$.
After discretization using the time step $\delta$,
we have the following system:
\begin{center}
\begin{gif}[][130][130]{convoyeqn2}\boldmath
$$
\begin{array}{rlll}
  1\leq i\leq N-1 : & a_i' &=& \lambda \mathit{dv}_i +
  \lambda_2(\mathit{dp}_i - 50) \\
  0\leq i\leq N-2 :& \mathit{da}_i' &=& a_i'-a_{i+1}' \\
  0\leq i\leq N-2 :& \mathit{dv}_i' &=& \mathit{dv}_i +
  \delta\cdot\mathit{da}_i \\
  0\leq i\leq N-2 :& \mathit{dp}_i' &=& \mathit{dp}_i +
  \delta\cdot\mathit{dv}_i +
  \delta^2/2\cdot\mathit{da}_i \\
  & v_{0}' &=& v_{0} + \delta\cdot a_{0} \\
  \multicolumn{4}{l}{\textrm{we take the constants}} \\
  & \lambda &=& 0.05 \\
  & \lambda_2 &=& 0.05 \\
  & \delta = 0.1
\end{array}
$$
\end{gif}
\end{center}
The purpose of the analysis is to show that there is no collision,
and to compute bounds on the relative positions and speeds of the
cars.

In experiments, we will use a constant acceleration for the
leading car inside an interval, and we will constrain other
initial values by intervals as well.

\subsection{Structure of the real Jordan form}
This example is interesting as the real Jordan form of the loop
body has blocks associated to complex eigenvalues of size $N-1$,
where $N$ is the number of cars.

With a constant acceleraton for the leading car, for $N=2$ we have
the real Jordan form
\begin{center}
\begin{gif}[][130][130]{convoyeqn3}\boldmath

  $
 \begin{pmatrix}
    \begin{pmatrix}
      0.997 & -0.022 \\
      0.022 & 0.997
    \end{pmatrix} & 0 & 0 & 0 \\
    0 & \begin{pmatrix}0.0047\end{pmatrix} & 0 & 0\\
    0&0&
    \begin{pmatrix}
      1 & 1 \\
      0 & 1
    \end{pmatrix} \\
    0&0&0& \begin{pmatrix}1\end{pmatrix}
  \end{pmatrix}
  $
\end{gif}
\end{center}
(the 6th dimension corresponds to the constant dimension added to
model the affine transformation with a linear one).

For $N=3$ we have
the real Jordan form
\begin{center}
\begin{gif}[][130][130]{convoyeqn4}\boldmath
  $
  \begin{pmatrix}
    \begin{pmatrix}
      0.997 & -0.022 & 1 & 0 \\
      0.022 & 0.997 & 0 & 1  \\
      0 & 0 & 0.997 & -0.022  \\
      0 & 0 & 0.022 & 0.997
    \end{pmatrix} & 0 & 0 & 0\\
    0 & \begin{pmatrix}
      0.0047 & 1 \\
      0 & 0.0047
    \end{pmatrix} & 0 & 0 \\
    0&0&
    \begin{pmatrix}
      1 & 1 \\
      0 & 1
    \end{pmatrix} \\
    0&0&0& \begin{pmatrix}1\end{pmatrix}
  \end{pmatrix}
  $
\end{gif}
\end{center}

In practice, this corresponds to the common experience that with
such a control law, the farther is a car from the leading car, the
larger will be its oscillations (in relative position and speed).

\subsection{Analysis}

\begin{center}
\begin{gif}[][130][130]{convoyeqn5}\boldmath
\begin{tabular}{|l|p{10em}|rrr|p{10em}|}
\hline
& initial condition & $\ell$ & \#e & running time & result \\ \hline
2 cars &
$a_0\in[0,1]$, $v_0\geq1$, $\mathit{dp}_0\in[45,55]$
$\mathit{dv}_0\in[-1,1]$, $\mathit{da}_0\in[-0.128,0.128]$ &
0 & 5 & 0.39 &
$a_0 \in [0,1]$, $v_0\geq1$, $\mathit{dp}_0\in[38.9,93.1]$
$\mathit{dv}_0\in[-4.69,5.78]$, $
\mathit{da}_0\in[-1.21,1.434]$ \\
-- & -- & 1 & 25 & 1.95 &
$a_0\in[0,1]$, $v_0\geq1$, $\mathit{dp}_0\in[42.7,90.3]$
$\mathit{dv}_0\in[-4.05,5.25]$, $\mathit{da}_0\in[-1.04,1.33]$ \\
-- & -- & 2 & 65 & 10.35 &
$a_0\in[0,1]$, $v_0\geq1$, $\mathit{dp}_0\in[43.3,89.0]$
 $\mathit{dv}_0\in[-3.72,5.03]$, $\mathit{da}_0\in[-0.98,1.31]$ \\
\hline
3 cars &
$a_0=0$, $v_0\geq1$, $\mathit{dp}_i\in[45,55]$
$\mathit{dv}_i=0$, $\mathit{da}_i=0$ &
0 & 8 & 19.9 &
$a_0=0$, $v_0\geq1$, $\mathit{dp}_1\in[33.5,66.49]$, $\mathit{dp}_0\in[44.6,55.41]$,
$\mathit{dv}_1\in[-4.05,4.05]$, $\mathit{dv}_0\in[-0.97,0.97]$,
$\mathit{da}_1\in[-1.17,1.17]$, $\mathit{da}_0\in[-0.27,0.27]$ \\
-- &
$a_0=0$, $v_0\geq1$, $\mathit{dp}_i=50$
$\mathit{dv}_i\in[-1,1]$, $\mathit{da}_i=0$ &
0 & 8 & 41.7 &
$a_0=0$, $v_0\geq1$, $\mathit{dp}_1\in[33.6,66.3]$, $\mathit{dp}_0\in[46.0,54.0]$,
$\mathit{dv}_1\in[-3.61,3.61]$, $\mathit{dv}_0\in[-1.08,1.08]$,
$\mathit{da}_1\in[-0.96,0.96]$, $\mathit{da}_0\in[-0.24,0.24]$ \\
-- &
$a_0=0$, $v_0\geq1$, $\mathit{dp}_i\in[45,55]$
$\mathit{dv}_i\in[-1,1]$, $\mathit{da}_i=0$ &
0 & 8 & 2134 &
$a_0=0$, $v_0\geq1$, $\mathit{dp}_1\in[20.2,79.8]$, $\mathit{dp}_0\in[40.6,59.4]$,
$\mathit{dv}_1\in[-6.50,6.50]$, $\mathit{dv}_0\in[-1.87,1.87]$,
$\mathit{da}_1\in[-1.72,1.72]$, $\mathit{da}_0\in[-0.47,0.47]$ \\
-- &
$a_0=0.256$, $v_0\geq1$, $\mathit{dp}_i\in[45,55]$
$\mathit{dv}_i\in[-1,1]$, $\mathit{da}_1=0$, $\mathit{da}_0=0.128$ &
0 & 8 & 199.9 &
$a_0=0.256$, $v_0\geq1$, $\mathit{dp}_1\in[15.7,92.7]$, $\mathit{dp}_0\in[40.2,66.1]$,
$\mathit{dv}_1\in[-7.39,7.43]$, $\mathit{dv}_0\in[-2.52,2.88]$,
$\mathit{da}_1\in[-2.05,1.94]$, $\mathit{da}_0\in[-0.58,0.70]$ \\
\hline
\end{tabular}
\end{gif}
\end{center}

Some comments:
\begin{itemize}
\item We succeed to compute lower and upper bounds on the state
  variables of system, for an initial condition which much more
  general than a single point.
\item Here the number of dimensions begins is higher, and we have
  bounded rotating behavior. This generates ``ellipsoidal''
  polyhedra with many constraints and generators, hence running
  times are significanlty higher than in other examples.
\item The good news is that even with the paramater ``l'' set to
  zero (that is, we just compute lower and upper bounds on the
  coefficients of integer powers of the Jordan normal form), we
  still obtain interesting bound.
\item As expected, the variations of relative position and speed of
  car 0 are higher than those of car 1 (and car 2 for the 3-cars examples).
\end{itemize}


%******************************************************************************
\section{All Experiments and Comparisons}
\label{sec:all}
%******************************************************************************

\begin{tabular}{lclrccccc}
name            & CFG         & type                     & nbvars
& Kleene & Jordan vs Kleene & Kleene time & Jordan time &
Ellipsoid applicable \\
\xlink{parabola\_i1}{parabola\_i1.lts}& single loops & unstable with guard      & 3
& 14/60  & +8, +17 & 0.014 & 0.007 & no \\
\xlink{parabola\_i2}{parabola\_i2.lts}& single loops & unstable with guard      & 3
& 24/60  & +18, +6 & 0.016 & 0.008 & no \\
\xlink{cubic\_i1}{cubic\_i1.lts}& single loops & unstable with guard      & 4
& 26/80  & +18, +26 & 0.077 & 0.013 & no \\
\xlink{cubic\_i2}{cubic\_i2.lts}& single loops & unstable with guard      & 4
& 46/80  & +38, +9, -2 & 0.086 & 0.012 & no \\
\xlink{exp\_div}{exp\_div.lts}& single loops & unstable with guard      & 2
& 6/24   & +2, +6, -1 & 0.007 & 0.004 & no \\
\xlink{oscillator\_i0}{oscillator\_i0.lts}& single loop & both types without guard & 2
& 27/28  & +23, +0, -1 & 0.013 & 0.004 & yes \\
\xlink{oscillator\_i1}{oscillator\_i1.lts}& single loop & both types without guard & 2
& 28/28  & +23, +0, -1 & 0.013 & 0.004 & yes \\
\xlink{inv\_pendulum}{inv\_pendulum.lts}& single loop & stable without guard     & 4
& 36/40 & +36,0 +0 & 0.538 & 0.009  & yes \\
\xlink{thermostat}{thermostat.lts}& nested loops & stable with guard        & 3
& 6/24 & +6, +1, -1 & 0.012 & 0.003 & no \\
\xlink{oscillator2\_16}{oscillator2\_16.lts}& nested loops & stable and unstable with guard & 3
& 39/48 & +39, +0 & 0.141 & 0.003 & no \\
\xlink{oscillator2\_32}{oscillator2\_32.lts}& nested loops & stable and unstable with guard & 3
& 39/48 & +39, +0 & 0.355 & 0.003 & no \\
\end{tabular}

We took templates of the form
  \begin{center}
    \begin{gif}[][130][130]{eqnexperi2}
      $$
      \alpha\,\epsilon_i m_i \sadd (1\ssub \alpha)\epsilon_j m_j \textrm{~with~}
      \alpha\seq\frac{k}{2^l}, 0\sleq k\sleq 2^l,
      \epsilon_i,\epsilon_j\in\{-1,1\}
      $$
   \end{gif}
  \end{center}
 where the $m$'s are the coefficients of the powers of the Jordan normal
 form, and $l$ is a parameter, set to $l=2$ for most experiments,
 except \textbf{oscillator2\_xx} that need $l=3$ to show finite
 bounds everywhere, as well waiting from the 3rd iteration before
 applying widening (needed because of the outer loop).

About the numbers: we take the bounding boxes of the inferred
polyhedra at each control point (except initial ones), and we
measure the improvements on bounds.
\begin{itemize}
\item ``Kleene'': nb of infinite bounds / total nb of bounds;
\item ``Jordan vs Kleene'': nb of bounds becoming finite, nb of
  improved finite bounds [, nb of worse finite bounds] (can be
  reduced to 0 by increasing $l$ to $3$ or $4$).
\item ``Kleene time'' and ``Jordan time'': times in seconds.
\end{itemize}

About the examples:
\begin{itemize}
\item \textbf{parabola\_i1} and \textbf{parabola\_i2}: loops of the
  type
  \begin{quote}\tt
    t=0; while (guard)\{ x=x+y; y=y+1; t=t+1; \}
  \end{quote}
  with
  different formula for \texttt{guard}, presented in
  \ref{sec:parabola}. Trajectories are discretized parabola.

  \textbf{i1} refers to the initial set $x,y ~in~ [0,2]$, \textbf{i2}
  to $x,y ~in~ [-2,2]$, which makes the evolution of \texttt{x}
  non-monotonic.
\item \textbf{cubic\_i1} and \textbf{cubic\_i2}: loops of the
  type
  \begin{quote}\tt
    t=0; while (guard)\{ x=x+y; y=y+z; z=z+1; t=t+1; \}
  \end{quote}
  with
  different formula for \texttt{guard}.

  \textbf{i1} refers to the initial set $x,y,z ~in~[0,2]$, \textbf{i2}
  to $x,y,z ~in~[-2,2]$, which makes the evolution of \texttt{x} and \texttt{y}
  non-monotonic.
\item \textbf{oscillator\_i0} and \textbf{oscillator\_i1}
  implement the (naive) discretization of the
  differential equation
  \begin{center}
    \begin{gif}[][130][130]{eqnexperi3}
      $$
      \ddot{\theta} + 2k\dot{\theta} + \omega^2
      \theta=0
      $$
    \end{gif}
  \end{center}
  with $w=0.125$ and different values for $k$, presented in
  \ref{sec:parabola}. Trajectories are inward (or outward int the
  unstable case) spirals.

  \textbf{i0} refers to the initial set
  $\theta(0)=8, \mathrm{d}\theta/\mathrm{d}t(0)=0$, \textbf{i1}
  to $\theta(0)=8, \mathrm{d}\theta/\mathrm{d}t(0)=8$.
\item \textbf{inv\_pendulum} is a PID-controlled inverted
  pendulum, with eigenvalues very close to 1.
\item \textbf{thermostat} is the thermostat example discussed in
  \ref{sec:thermostat}. Trajectories are successions of pieces of
  exponentials.

\item \textbf{oscillator2\_16} and \textbf{oscillator2\_32} are the
  damped oscillator with three modes discussed in
  \ref{sec:hoscillator}. \textbf{16} (resp. \textbf{32})
  corresponds to a damping factor of $1/16$ (resp. $1/32$) of the
  central part/mode.  Trajectories are successions of pieces of
  inward/outward spirals.  In the case of \textbf{oscillator2\_16},
  the damping is strong enough and the analysis precise enough to
  show that the mobile point cannot pass a second time in the
  central part/mode from a negative position.
\end{itemize}

If we now set the template parameter $l$ to $l=4$ for all
examples, to give an idea of the improvement in precision
w.r.t. standard analysis and the increase in runing times, we get
the following table:

\begin{tabular}{lclrccccc}
name            & CFG         & type                     & nbvars
& Kleene & Jordan vs Kleene & Kleene time & Jordan time &
Ellipsoid applicable \\
\xlink{parabola\_i1}{parabola\_i1.lts}& single loops & unstable with guard      & 3
& 14/60  & +8, +19 & 0.020 & 0.007 & no \\
\xlink{parabola\_i2}{parabola\_i2.lts}& single loops & unstable with guard      & 3
& 24/60  & +18, +9 & 0.022 & 0.008 & no \\
\xlink{cubic\_i1}{cubic\_i1.lts}& single loops & unstable with guard      & 4
& 26/80  & +18, +26 & 0.258 & 0.013 & no \\
\xlink{cubic\_i2}{cubic\_i2.lts}& single loops & unstable with guard      & 4
& 46/80  & +38, +11 & 0.524 & 0.012 & no \\
\xlink{exp\_div}{exp\_div.lts}& single loops & unstable with guard      & 2
& 6/24   & +2, +7 & 0.010 & 0.004 & no \\
\xlink{oscillator\_i0}{oscillator\_i0.lts}& single loop & both types without guard & 2
& 27/28  & +23, +0, -1 & 0.033 & 0.004 & yes \\
\xlink{oscillator\_i1}{oscillator\_i1.lts}& single loop & both types without guard & 2
& 28/28  & +24, +0 & 0.033 & 0.004 & yes \\
\xlink{inv\_pendulum}{inv\_pendulum.lts}& single loop & stable without guard     & 4
& 36/40 & +36,+0 & 96.229 & 0.012 & yes \\
\xlink{thermostat}{thermostat.lts}& nested loops & stable with guard        & 3
& 6/24 & +6, +1, -1 & 0.017 & 0.003 & no \\
\xlink{oscillator2\_16}{oscillator2\_16.lts}& nested loops & stable and unstable with guard & 3
& 39/48 & +39, +0 & 0.300 & 0.003 & no \\
\xlink{oscillator2\_32}{oscillator2\_32.lts}& nested loops & stable and unstable with guard & 3
& 39/48 & +39, +0 & 0.706 & 0.003 & no \\
\end{tabular}
The increase in running time become moderate in general, less than the
worst-case if one look to the increase of the number of template
constraints. The exception is the inverted pendulum example.




\end{document}
